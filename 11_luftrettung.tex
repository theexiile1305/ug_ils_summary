\section{Luftrettung}
\begin{sectionbox}{Leitfaden Einsatz von RTH nach BayStmI}
    \begin{itemize}
        \item Entscheidung obliegt zuständiger ILS\\
        \ra Berücksichtigung der Einsatzkriterien
        \item Berücksichtigung von landesrechtlichen Vorgaben\\
        \ra Richtlinien über Zusammenarbeit von Luftrettungsmitteln mit anderen Einsatzkräften
        \item Primäreinsatz\\
        \ra Notarzindikation erforderlich sowie medizinischer Zeitvorteil gegenüber bodengebundenen Notarzt\\
        \ra Wasserrettung, schwierig erreichbare Notfallstelle\\
        \ra Hilfsfrist des bodengebundenen RD nicht möglich
        \item Sekundäreinsatz\\
        \ra bei RTH möglich, falls Primärauftrag beachtet\\
        \ra Verfügbarkeit andere RM incl. ITH einbeziehen
        \item Transport Medikamte, Blutkonserven, Transplantate\\
        \ra RTH muss ultima ratio sein \& intensive Prüfung 
    \end{itemize}
\end{sectionbox}
\begin{sectionbox}{Flotte}
    \begin{itemize}
        \item Notarzt: mehrjährige Erfahrung \& Facharztstandard
        \item RA/NS: jährliche Auffrischung \& HCM
        \item Pilot: entscheidet Einsatzdurchführbarkeit
    \end{itemize}
\end{sectionbox}
\begin{sectionbox}{Vorteile}
    \begin{itemize}
        \item schnellstmöglicher, schonender Transport
        \item entfernte Fachkliniken \ra Interhospitaltransfer
        \item Suchflüge, unwegsames Gelände \& Ergänzung Boden
    \end{itemize}
\end{sectionbox}
\begin{sectionbox}{Grenzen}
    \begin{itemize}
        \item Wasser, Dunkelheit, Infektionen
        \item Technik, Einsatzzeit (07:00 -- Sunset)
    \end{itemize}
\end{sectionbox}