\section{Feuerwehr}
\subsection{Bayerisches Feuerwehrgesetz (BayFwG)}
\begin{normbox}{\subsubsection{Allgemeines}}
    \begin{itemize}
        \item Grundlage für Arbeit der Feuerwehr\\
        \ra zusätzliche Beachtung: Ausführungsverordnung \& Vollzugsbekanntmachung
        \item 20.12.11': aktuelle Fassung
        \item mitgeltende Grundlagen: ILSG \& ABek
    \end{itemize}
\end{normbox}
\begin{normbox}{\subsubsection{I - Aufgaben \& Träger}}
    \begin{itemize}
        \item Gemeinde: Ausrüstung \& Unterhalt Feuerwehr\\
        \ra abwehrender u. vorbeugender Brandschutz\\
        \ra Löschwasserversorgungsanlagen\\
        \ra technischer Hilfsdienst
        \item Landkreise: Beschaffungen, Unterhalt \& Zuschüsse\\
        \ra überörtliche Geräte \& Einrichtungen
        \item Staat: Förderungen \& Zuwendungen\\
        \ra Unterhalt Feuerwehrschule \& Betrieb ILSG 
    \end{itemize}
\end{normbox}
\begin{normbox}{\subsubsection{II - Die Feuerwehren}}
    \begin{itemize}
        \item ehrenamtliches Personal beim Dienst 
        \item Zusammenarbeit der Feuerwehren einer Gemeinde
        \item Verpflichtung zur überörtlichen Hilfe
        \item Regelungen zur Einsatzleitung
    \end{itemize}
\end{normbox}
\begin{normbox}{\subsubsection{III - Besondere Führungsdienstgrade}}
    \begin{itemize}
        \item Kreisbrandrat untersteht Landrat\\
        \ra berät \& unterstützt Landkreis \& Kommune
        \item Verbände: Interessenvertretung der Feuerwehren\\
    \end{itemize}
\end{normbox}
\begin{normbox}{\subsubsection{IV - Pflichten der Bevölkerung}}
    \begin{itemize}
        \item Heranziehung von Personen \& Sachen
        \ra nur Einsatzleiter, Entschädigunganspruch!\\
        \item Verhältnismäßigkeit muss zwingend gegeben sein
        \item Platzverweis, falls POL nicht zur Verfügung steht
    \end{itemize}
\end{normbox}
\begin{normbox}{\subsubsection{V - Kosten \& Schlussvorschriften}}
    \begin{itemize}
        \item kostenlos, falls Rettung/Bergung Menschen/Tiere\\
        \ra Finanzierung durch Feuerschutzsteuer
        \item Schlussvorschriften: StMI kann RV \& DVO erlassen
    \end{itemize}
\end{normbox}
\begin{normbox}{\subsubsection{Ausführungsverordnung BayFwG}}
    \begin{itemize}
        \item ergänzende Festlegungen zum Gesetz
        \item beschreibt organisatorischen Aufbau
        \item regelt Ausbildung des Disp. der ILS
    \end{itemize}
\end{normbox}
\begin{normbox}{\subsubsection{Vollzugsbekanntmachung BayFwG}}
    \begin{itemize}
        \item zusätzliche Hinweise \& Erläuterungen\\
        \ra zur Rechtslage \& Empfehlungen
        \item Hilfsfrist der Feuerwehren
        \item Mustersatzungen FF/Vereine \& Musterkostensatzung
        \item informiert Arbeitgeber über Erstattungen
    \end{itemize}
\end{normbox}
\begin{sectionbox}{Aufgaben der Feuerwehr}
    \begin{itemize}
        \item Pflichtaufgaben: abwehrender Brandschutz\\
        \ra technischer Hilfsdienst \& Katastrophenhilfe\\
        \ra Brandwache, Amtshilfe \& Sicherheitswachen
        \item freiwillige Tätigkeiten: falls Einsatzbereitschaft\\
        \ra kann durch Kdt. oder Führung angeordnet werden\\
        \ra ggf. Einwilligung der Gemeinde\\
        \ra keine Konkurrenz gegenüber Privatwirtschaft 
    \end{itemize}
\end{sectionbox}
\begin{hintbox}{Rufnamen Führungsdienstgrade}
    \begin{itemize}
        \item Kommandant: FL (Ort) 1
        \item Kreis-/Stadtbrandrat: FL (Lkr./Stadt) 1
        \item Kreis-/Stadtbrandinspektor: FL (Lkr./Stadt) 2/1 - 2/9
        \item  Kreis-/Stadtbrandmeister: FL (Lkr./Stadt) 3/1 - 3/9
    \end{itemize}
\end{hintbox}
\subsection{Weisungsrecht}
\begin{normbox}{\subsubsection{Kraft Gesetz}}
    \begin{itemize}
        \item grundsätzlich Kdt. des Einsatzortes
        \item Leiter Einsatzkräfte BF des Schadensortes
        \item ranghöchster Einheitsführer der FF des Schadensort
        \item Leiter der WF
    \end{itemize}
\end{normbox}
\begin{normbox}{\subsubsection{Kraft Übernahme}}
    \begin{itemize}
        \item Kdt. der gemeindlichen FF mit überwiegenden Einsatzmitteln
        \item Leiter von Einsatzkräften einer FF in Betrieben/WF
        \item besondere Führungsdienstgrade
        \item Übernahmerecht aufgrund Übertragung durch KBR
        \item Leiter von Einsatzkräften (gD, hD) der BF
        \item besonders qualifizierte Leiter von Einsatzkräften ständiger Wache
    \end{itemize}
\end{normbox}
\begin{normbox}{\subsubsection{Kraft Übertragung}}
    \begin{itemize}
        \item geeignete Person aufgrund Übertragung durch KBR
        \item besondere Führungsdienstgrade bei zeitgleichen Einsätzen im Zuständigkeitsbereich des Landkreises
        \item Weisungsbefugnis gegenüber Einsatzleitern \& KEZ
    \end{itemize}
\end{normbox}
\begin{normbox}{\subsubsection{Kraft Gesetze/Verordnungen}}
    \begin{itemize}
        \item Regierung: besonderes gefährdetes Objekt über ein Gebiet mehrerer Landkreise
        \item Bundesbedienstete: Schadenstelle bei bundeseigener Verwaltung \ra Landkreis überträgt Einsatzleitung 
        \item Bergamt in Bergbaubetrieben nach Bundesberggesetz
    \end{itemize}
\end{normbox}
\begin{warningbox}{Einsatzleitung vor Ort}
    Mitteilung ILS \& Sicherstellen der Erreichbarkeit
\end{warningbox}
\begin{hintbox}{Übersicht Einsatzmittel}
    \begin{itemize}
        \item 10-19: Führungsfahrzeuge und Einsatzleitwagen
        \item 20-29: Tanklöschfahrzeuge
        \item 30-39: Hubrettungsfahrzeuge
        \item 40-49: Löschgruppen- und Tragkraftspritzenfahrzeuge
        \item 50-59: Geräte- / Transportfahrzeuge
        \item 60-69: Rüst- und Gerätewagen
        \item 70-79: Rettungs- und Sanitätsfahrzeuge
        \item 80-89: Bergrettungsfahrzeuge
        \item 90-99: Wasserrettungsfahrzeuge
    \end{itemize}
\end{hintbox}
\subsection{Führungsstufen}
\begin{normbox}{\subsubsection{Führungsstufe A}}
    \begin{itemize}
        \item Führen ohne Führungseinheit\\
        \ra taktische Einheiten: bis zu 2 Gruppen\\
        \ra Führungseinrichtung: ILS
        \item Beispiel\\
        \ra FW: Staffel-/Gruppenlage ohne Führungsassistent\\
        \ra RD: ELRD ohne Führungsassistent
    \end{itemize}
\end{normbox}
\begin{normbox}{\subsubsection{Führungsstufe B}}
    \begin{itemize}
        \item Führen mit örtlicher Führungseinheit\\
        \ra Zug/Verband an Einsatzstelle\\
        \ra Führungstrupp oder -staffel
        \ra Führungseinrichtung: ILS
        \item Beispiel\\
        \ra FW: Zuglage (ZF mit Zugtrupp)\\
        \ra RD: ELRD mit ELW
    \end{itemize}
\end{normbox}
\begin{normbox}{\subsubsection{Führungsstufe C}}
    \begin{itemize}
        \item Führen mit einer Führungsgruppe\\
        \ra Verband an Einsatzstelle\\
        \ra Führungsgruppe\\
        \ra Führungseinrichtung: ILS
        \item Beispiel\\
        \ra FW: Verbandführer (VF mit UG FwEL)\\
        \ra RD: SanEL (OrgL/LNA mit UG SanEL)\\
        \ra K-Einsatz: ÖEL-Lage (ÖEL mit UG ÖEL)
    \end{itemize}
\end{normbox}
\begin{normbox}{\subsubsection{Führungsstufe D}}
    \begin{itemize}
        \item Führen mit einer Führungsgruppe oder -stab\\
        \ra mehrere Verbände an Einsatzstelle oder mehrere Einsatzstellen im Schadensgebiet\\
        \ra Führungsgruppe bzw. -stab des LK
        \ra Führungseinrichtung: ILS oder IuK-Zentrale
        \item Beispiel\\
        \ra K-Einsatz: FüGK (FüGK mit KomFü)
    \end{itemize}
\end{normbox}