\section{Giftnotruf}
\begin{sectionbox}{Aufgaben}
    \begin{itemize}
        \item Sicherstellungsauftrag Erstberatung \& Informationen\\
        \ra Verwendung von Laien \& professionelle Personen
        \item Klärung von Fragen zu Giften, Drogen, Chemikalien
    \end{itemize}
\end{sectionbox}
\begin{sectionbox}{Erreichbarkeit}
    \begin{itemize}
        \item Träger: Bundesländer, auch im Katastrophenfall 24/7\\
        \ra Ortsvorwahl + 19 240
        \item Rufbereitschaft, auch für Großeinsätze als auch Politik
        \ra kompetente Führungsposition
    \end{itemize}
\end{sectionbox}
\begin{sectionbox}{Datenbanken}
    \begin{itemize}
        \item Beratung: umfassendes Informationsmaterial
        \item nationale \& internationale Datenbanken\\
        \ra z.B. München: selbst erstellte Toxinfo DB 
    \end{itemize}
\end{sectionbox}
\begin{warningbox}{Fragen zur Informationsgewinnung}
    \begin{itemize}
        \item Was wurde ein-/aufgenommen?
        \item Wie wurde der Stoff ein-/aufgenommen?
        \item Wieviel wurde ein-/aufgenommen?
        \item Wann wurde der Stoff ein-/aufgenommen?
        \item Wie sind die Vitalparameter?
        \item Zusätzliche Fragen wie Alter und Körpergewicht
    \end{itemize}
\end{warningbox}