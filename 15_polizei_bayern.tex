\section{Bayrische Landespolizei}
\begin{sectionbox}{Organisation}
    \begin{itemize}
        \item definiert im Polizeiorganisationsgesetz (POG)
        \item Präsidialbereich: untergliedert in Inspektionsbereiche
        \item jede PI nimmt Aufgaben grundsätzlich in ihrem Bereich vor Ort wahr
    \end{itemize}
\end{sectionbox}
\begin{sectionbox}{Einsatzzentrale der Polizei}
    \begin{itemize}
        \item Aufgabe: zentrale Steuerung, Koordinierung \& Führung aller Einsätze im Präsidialbereich
        \item Übernahme der Aufgaben durch Führungsstab bei herausragenden Lagen
    \end{itemize}
\end{sectionbox}
\begin{sectionbox}{Gesetzliche Grundlagen}
    \begin{itemize}
        \item Grundlage der Aufgaben bildet Art. 2 (1-4) PAG\\
        \ra Gefahrenabwehr gemäß PAG o. weitere Gesetze
        \item PAG regelt Verhältnis zu alen BOS
        \item StPO: Verfolgung von Straftaten \& Ordnungswidrigkeiten
        \item Übertragung von Befugnissen\\
        \ra z.B. freiheitsbeschränkende Maßnahmen
    \end{itemize}
\end{sectionbox}
\begin{sectionbox}{Zusammenarbeit mit andern BOS}
    \begin{itemize}
        \item zahlreiche Berührungspunkte mit expliziten Rechten
        \item Problemfelder: bei Großlagen, aus Unkenntnis, Unzulänglichkeiten o. unzureichende Kommunikation
    \end{itemize}
\end{sectionbox}