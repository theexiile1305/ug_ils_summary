\section{Bayrisches Rettungsdienstgesetz}
\begin{sectionbox}{(2) Begriffsbestimmungen (1)}
    \begin{itemize}
        \item öffentlicher RD: alle Beteiligten des ZRF \& KVB
        \item Notfallrettung: notfallmedizinische Versorgung von Notfallpatienten \& Notfalltransport
        \item Notfallpatienten: Verletzte/Kranke in Lebensgefahr oder schwere gesundheitliche Schäden, falls nicht erforderlich medizinisch Versorgung erfolgt 
        \item notfallmedizinische Versorgung: Abwendung von Lebensgefahr \& schweren gesundheitlichen Schäden\\
        \ra Herstellung Transportfähigkeit von Notfallpatienten
        \item Notfalltransport: Beförderung von Notfallpatienten unter fachgerechter medizinischer Betreuung in eine für die weitere Versorgung geeignete Einrichtung
        \item Notarztdienst: Notärzte in Notfallrettung
        \item Notärzte: Notarztqualifikation als besondere medizinische Kenntnisse \& Fähigkeiten für die Behandlung \& Transport von Notfallpatienten
        \item Arztbegleiteter Patiententransport: Beförderung von Patienten mit medizinischer Betreuung/Überwachung durch Verlegungsarzt/geeigneter Krankenhausarzt
        \item Verlegungsärzte: Ärzte mit besondere Kenntnisse, Fähigkeiten \& Fertigkeiten
        \item Krankentransport: Transport von !Notfallpatienten mit medizinischer Betreuung durch nichtärztliches Fachpersonal oder KTW bedürfen oder zu erwarten ist
    \end{itemize}
\end{sectionbox}
\begin{sectionbox}{(2) Begriffsbestimmungen (2)}
    \begin{itemize}
        \item Luftrettung: Durchführung von Notfallrettung, arztbegleitetem Patiententransport, Unterstützung von Einsätzen der Landrettung, der Berg- und Höhlenrettung \& Wasserrettung
        \item RTH: primäre für Notfallrettung
        \item ITH: primärer für arztbegleiteten Patiententransport 
        \item Organisierte Erste Hilfe: nachhaltig, planmäßig \& auf Dauer stattfinde Erste Hilfe am Notfallort\\
        \ra kein Bestandteil des öffentlichen RD
        \item Sanitätsdienst: im Auftrag erfolgende medizinische Absicherung von Veranstaltungen \& medizinische Betreuung von Patienten \ra kein Transport
    \end{itemize}
\end{sectionbox}
\begin{sectionbox}{(4) Aufgabenträger}
    \begin{itemize}
        \item oberste RD-Behörde: Lankdreise \& kreisfreien Städte
        \item RD untergliedert in RD-Bereiche\\
        \ra Aufgabe: Angelegenheit des übertragenen Wirkungskreises\\
        \ra Sicherstellung von Effektivität \& Wirtschaftlichkeit
        \item Zusammenschluss als ZRF
    \end{itemize}
\end{sectionbox}
\begin{sectionbox}{(5) Aufgaben der Aufgabenträger}
    \begin{itemize}
        \item ZRF: Sicherstellung RD durch Versorgungsstrukturen
        \item betroffene ZRFs müssen informiert werden, falls Entscheidungen anderer RD-Bereiche betreffen
        \item Vereinbarung von Versorgungsplanungen bei ortsübergreifender Gebieten durch ZRFs
    \end{itemize}
\end{sectionbox}
\begin{sectionbox}{(7) Einrichtungen des öffentlichen RD}
    \begin{itemize}
        \item ILS, ärztlicher Leiter \& ganztätig besetzt RW/NS\\
        \ra Standorte für Verlegungsärzte, Stellplätze oder andere RW sind möglich
        \item Gewährleistung der Einhaltung der Hilfsfrist
    \end{itemize}
\end{sectionbox}
\begin{sectionbox}{(9) Einsaztlenkung des öffentlichen RD}
    \begin{itemize}
        \item Lenkung \& Abstimmung Einsätze durch ILS
        \item fachliche Vorgaben/Gesetze sind stets zu beachten
    \end{itemize}
\end{sectionbox}
\begin{sectionbox}{(14) Notarztdienst}
    \begin{itemize}
        \item ZRF \& KVB: Sicherstellung Notärzte bei Einsätzen
        \item Weisungsrecht des Notarztes gegenüber Einsaztkräfte 
    \end{itemize}
\end{sectionbox}
\begin{sectionbox}{(15) Arztbegleiteter Patiententransport}
    VEF/ITW durch den für ihre zuständigen ILS eingesetzt, solange oberste RD-Behörde nicht anders bestimmt
\end{sectionbox}
\begin{sectionbox}{(17) Berg- und Höhlenrettung}
    ZRF überträgt Durchführung Berg- und Höhlenrettung an Bergwacht oder im Rahmen geeigneter Auswahlverfahren an private Unternehmen.
\end{sectionbox}
\begin{sectionbox}{(18) Wasserrettung}
    ZRF überträgt Durchführung Wasserrettung an Wasserwacht \& DLRG oder im Rahmen geeigneter Auswahlverfahren an private Unternehmen.
\end{sectionbox}
\begin{sectionbox}{(19) Sonderbedarf bei Großschadenslagen}
    \begin{itemize}
        \item besonderes Schadensereignis\\
        \ra erfordert besondere Vorgehensweise des RD\\
        \ra ebenso besondere Koordinierung mit Kräften des Sanitäts- oder Betreuungsdienstes
        \item SanEL erweitert ELRD bei besonderen Schadensereignissen
        \item SanEL besteht aus OrgL \& LNA
    \end{itemize}
\end{sectionbox}
\begin{sectionbox}{(43) Besetzung, Personalqualifikation}
    \begin{itemize}
        \item KTW: Besetzung mind. ein RS
        \item RTW: Besetzung mind. ein RA/NS
        \item NEF: Besetzung mind. ein NA\\
        \ra zusäztlich Fahrer, falls Beide vom gleichen Standort
        \item VEF: Besetzung mind. ein RA/NS
    \end{itemize}
\end{sectionbox}
\begin{sectionbox}{(53) Rechtsverordnungen \& Verwaltungsvorschriften}
    Oberste RD-Behörde kann Rechtsverordnungen treffen:
    \begin{itemize}
        \item Anforderungen an personeller Besetzung
        \item persönliche \& fachliche Voraussetzungen
        \item sächliche Ausstattung RD/RW
        \item Bemächtigung ZRF: NEF darf in begründeten Fällen auch alleine eingesetzt werden
    \end{itemize}
\end{sectionbox}