\section{Technisches Hilfswerk (THW)}
\begin{sectionbox}{Allgemeines}
    \begin{itemize}
        \item Bundesanstalt mit eigenen Geschäftsbereich des BMI
        \item Zivil- \& Katastrophenschutz auf Bundesebene
        \item örtliche Gefahrenabwehr auf Anforderung der Stellen
        \item Auslandseinsätze im Auftrag des Bundes
    \end{itemize}
\end{sectionbox}
\begin{sectionbox}{Alarmierung}
    \begin{itemize}
        \item Einbindung des Einsatzmittels, Fachgruppe und Fachberater durch Schlagwort
        \item gesonderte Alarmierung des Fachberaters möglich
        \item Alarmierung einzelner Fachgruppen auf Anforderung
    \end{itemize}
\end{sectionbox}
\begin{sectionbox}{Fachgruppen}
    \begin{itemize}
        \item Bergungsgruppe: Rettung Menschen/Tiere, Bergungen\\
        \ra Sicherungsarbeiten, Räumarbeiten, Wiederherstellung Wege/Übergänge
        \item Infrastruktur: Elektro-, Wasser-, Abwassersysteme\\
        \ra Stromversorgung von Einsatzgeräten
        \item Räumen: Beseitigung Hindernisse/Trümmer
        \item Elektroversorgung: temporäre Stromversorgung\\
        \ra zuständig für Unterkünfte, Kommune, KRITIS
        \item Wasserschaden/Pumpe: Überflutungen
        \item Ölschaden: Bekämpfung von Schadstoffaustritten
        \item Brückenbau: kurzfristige Behelfsbrücken\\
        \ra 50m Länge \& 120m Eisenbahn-Brücke
        \item Ortung: technisches Ortungsgerät
        \item Infrastruktur (FGrI): Gefahr Elektrizität, Wasser, Gas\\
        \ra provisorische Instandsetzung Gebäudeinstallation
        \item Sprengen: z.B. Sprengen von Eisscholen an Brücken
        \item Beleuchtung: breite Palette an Beleuchtunsgeräten
        \item Logistik: Materialerhaltung \& Verpflegung
        \item Trinkwassserversorgung: Wasseraufbereitungsanlagen mit Wasseranalysen
        \item Führung/Kommunikation: Unterstützung Einsatzleitung
    \end{itemize}
\end{sectionbox}
\begin{sectionbox}{Technische Züge}
    \begin{itemize}
        \item besteht aus Zugtrupp, Bergungs- \& Fachgruppe
        \item Zuggruppe: Führung, Etablierung Führungsstelle\\
        \ra organisiert Logistik, Personal- \& Materialeinsatz
    \end{itemize}
\end{sectionbox}
\begin{warningbox}{THW Fachberater}
    \begin{itemize}
        \item vertritt Interessen auf Führung \& Koordinierung\\
        \ra Berücksichtigung von Anforderern \& Bedarfsträgern
        \item Beratung Einsatzleiter über Spektrum des THW\\
        \ra modulare Zusammenstellung für Potenziale\\
        \ra ständiges Mitglied im Führungstab
    \end{itemize}
\end{warningbox}
