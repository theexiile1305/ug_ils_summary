\section{RD Bayern: Aufbau \& Struktur}
\begin{sectionbox}{Aufbau des Rettungswesens}
    \begin{itemize}
        \item Notrufeinrichtungen
        \item Erste Hilfe als\\
        \ra spontane Laienhilfe\\
        \ra organisierte Erste Hilfe: FR, betrieblich
        \item Rettungsdienst\\
        \ra öffentlich: Notfallrettung, Krankentransport\\
        \ra privat: Krankentransport
        \item Krankenhausversorgung
    \end{itemize}
\end{sectionbox}
\begin{sectionbox}{Strukturen}
    \begin{itemize}
        \item Grundlage bildet BayRDG, AVBayRDG \& ILSG
        \item öffentlicher RD umfasst Notfallrettung, arztbegleiteter Patiententransport, Krankentransport, Berg- \& Höhlenrettung
        \item oberste RD-Behörde Bayerns: StmI
        \item Sicherstellung RD durch ZRF aller Landkreise \& landkreisfreien Gemeinden
        \item ärztlicher Leiter RD: Sicherstellung \& Verbesserung der Qualität des RD
    \end{itemize}
\end{sectionbox}
\begin{sectionbox}{Rettungswachen}
    \begin{itemize}
        \item Vorhaltung von Einsatzkräfte \& Rettungsmittel
        \item §1 AVBayRDG\\
        \ra (1): Hilfsfrist 12 min. bei Straße liegendem Einsatzort\\
        \ra (1): Hilfsfrist 15 min. bei dünn besiedelten Gebieten mit schwachen Verkehr\\
        \ra (2): Sicherstellung Versorgung durch Stellplatz
        \item §3 AVBayRDG\\
        \ra bei §1 (1) AVBayRDG Besetzung RTW 24/7
    \end{itemize}
\end{sectionbox}