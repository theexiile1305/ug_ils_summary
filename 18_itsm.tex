\section{Sicherheitskonzept der ILS}
\begin{warningbox}{Kritische Infrastrukturen (KRITIS)}
   \begin{itemize}
        \item Organisationen und Einrichtungen mit wichtiger Bedeutung für das staatliche Gemeinwesen
        \item Ausfall: Störungen öffentliche Sicherheit \& Versorgung
        \item laut BaySÜG Art. 3 (2/3) ist ILS KRITIS
   \end{itemize}
\end{warningbox}
\begin{sectionbox}{IT-Sicherheit}
    \begin{itemize}
        \item Sicherstellung des CIA-Principles
        \item \textbf{C}onfidentiality: Geheimhaltung von Informationen
        \item \textbf{I}ntegrity: Integrität von Informationen
        \item \textbf{A}vailability: Verfügbarkeit von Informationen
    \end{itemize}
\end{sectionbox}
\begin{sectionbox}{Bereiche}
    \begin{itemize}
        \item Personal: Sicherheitsüberprüfung Art. 3 (1/4) \& Schweigepflicht
        \item Gebäude: Einzäunung \& Videoüberwachung\\ 
        \ra Zutrittskontrolle via Token (anonym, Tracing)\\
        \ra stetige Beaufsichtigung von Besuchern\\
        \ra Fensterverglasung durchwurfsicher \& verspiegelt
        \item Technik: zwei getrennte Stromversorgungen\\
        \ra USV (2h), Notstromaggregat (12h)\\
        \ra externe Einspeisung durch THW\\
        \ra zwei getrennte Telefonleitungen\\
        \ra Ansteuerung Funk via VPN\\
        \ra Ausfall VPN: Ansteuerung via Funkmast
        \item EDV/Daten: Segmentierung der Netzwerke\\
        \ra Separation von Verwaltung \& Einsatzleitsystem\\
        \ra Authentifizierung, Authorisierung, Logging
    \end{itemize}
\end{sectionbox}