\section{Alarmierungsbekanntmachung ABEK}
\begin{sectionbox}{Allgemeines}
    \begin{itemize}
        \item flächendeckende ILS erfodert Anpassung Alarmierung\\
        \ra landesweit einheitliche Software \& Standards
        \item Zuständigkeiten\\
        \ra Brand/KatS: Kreisverwaltungsbehörde (KBI/ILS)\\
        \ra RD: ZRF (RD-Druchführende/ILS)
    \end{itemize}
\end{sectionbox}
\begin{sectionbox}{Ziele}
    \begin{itemize}
        \item schnelle, angemessene, ereignisbezogende Alarmierung
        \item flächendeckende, objekt- \& ereignisbezogende Planung
        \item landesweit einheitliche vorgegebene Einsatzstichwörter\\
        \ra Ergänzung durch unterschiedliche Module
        \item Zuordnung von Maßnahmen aufgrund von Zeiträumen, Zonen, Einsatzmittelketten \& Schlagwörtern
    \end{itemize}
\end{sectionbox}
\begin{sectionbox}{Vorteile durch Automatismus}
    \begin{itemize}
        \item Alarmierung relevanter, schnellster Einsatzmittel
        \item Ersatz bei Nichtverfügbarkeit von Einsatzmitteln
        \item Ermittlung Einsatzmittel bei Alarmstufenerhöhung
        \item klare Zuständigkeiten aller alarmierten Einheiten
    \end{itemize}
\end{sectionbox}
\begin{sectionbox}{System ELDIS 3 Bayern}
    \begin{itemize}
        \item Schlagwort impliziert Stichwörter für FF/RD/etc.\\
        \ra Erweiterung um spezifische Module möglich
        \item Einsatzstichwort + Schlagwort = Einsatzmittelkette
    \end{itemize}
\end{sectionbox}