\section{Konzept ILS Bayern}
\begin{sectionbox}{Notruf 112}
    \begin{itemize}
        \item 45': 110 (POL), 112 (Feuerwehr) \& 19222 (RD)
        \item Entwicklung\\
        \ra Anfang 90er: bundeseinheitliche Notrufnummer\\
        \ra 94': Landtagsbeschluss Prüfung 112 durch STMI
        \ra 96': Inbetriebnahme ILS München
        \item 29.07.1991: 112 europaweit \& einheitlich\\
        \ra große Bekanntheit \& echte Notrufnummer
    \end{itemize}
\end{sectionbox}
\begin{sectionbox}{Notruf- und Alarmierungsstrukturen}
    \begin{itemize}
        \item 97'-99': Vorgutachten für Bestandsaufnahme\\
        \ra RD: großräumige, weitgehend einheitlich\\
        \ra FF: kleinräumige, lokale Varianten
        \item 99'-00': Hauptgutachten von Lösungsmöglichkeiten\\
        \ra Einführung ILS \& einheitliche Notrufnummer 112\\
        \ra Beteiligung Verbände zu landesweiten Standards zu Aufgaben, Organisation, Personal, Qualifikation, Technik, Finanzierung
        \item 11.07.01': Einführung ILS
        \item 01.09.02': ILSG tritt in Kraft 
    \end{itemize}
\end{sectionbox}
\subsection{Einführung Integrierter Leitstellen (ILSG)}
\begin{normbox}{\subsubsection{Allgemeines}}
    \begin{itemize}
        \item gemeinsame Nutzung 112 (FW \& RD)
        \item ZRF: Zweckverband für Rettungsdienst und Feuerwehralarmierung
        \item Notrufabfrage \& Alarmierung durch ILS
        \item Wegfall Nachalarmierungsstellen \ra Bildung von KEZ
        \item Weisungsrecht gegenüber taktischen Fragen des RD
        \item Kostenübernahme durch Aufgabenbereiche RD/FF
        \item ausreichende Dokumentation des Einsatzes \& getroffene aufgabenbezogene Festellungen/Maßnahmen
    \end{itemize}
\end{normbox}
\begin{normbox}{\subsubsection{Aufgaben (Art. 2 ILSG)}}
    \begin{itemize}
        \item Entgegennahme aller Notrufe, Notfallmeldungen, sonstigen Hilfeersuchen \& Informationen
        \item Alarmierung erforderlicher Einsatzkräfte \& -mittel
        \item Begleitung aller Einsätze
        \item Unterstützung der Einsatzleitung
        \item Meldekopf der Kreisverwaltungsbehörde als Sicherheitsbehörde
        \item Führung über Krankenbettennachweis, Giftnotrufe, Not-Apotheken, Blutspenden \& Druckkammern
        \item Freiwillige Aufgaben: Vermittlung KVB, Alarmierung von weiteren Einheiten (Hvo, KIT, etc.)
    \end{itemize}
\end{normbox}
\begin{normbox}{\subsubsection{Überörtliche Zusammenarbeit (Art. 2 Abs. 6 ILSG)}}
    \begin{itemize}
        \item Zusammenarbeit: Nachbar-ILS, bet. Stellen/Kräfte\\
        \ra schnellstes verfügbares geeignetes Einsatzmittel
        \item Abstimmung Alarmierungsplanungen mit Nachbarn
    \end{itemize}
\end{normbox}
\begin{warningbox}{Betreiber ILS Oberland}
    direkt untergeordnet bei der Landesgeschäftsstelle BRK
\end{warningbox}
 \begin{sectionbox}{Kreiseinsatzzentralen (KEZ)}
    \begin{itemize}
        \item nach BayFwG Errichtung von ein/mehreren KEZ
        \item unterstützt in Abstimmung ILS Einsatzleiter
        \item keine Aufgabenübernahme der bisherigen NAST
        \item großflächige Schadensereignisse\\
        \ra selbstständige Bewältigung bestimmter Einsätze\\
        \ra Zuweisung erforderliche Einsatzkräfte \& Mittel\\
        \ra Entlastung ILS: ortsbezogene Einsatzübersicht
        \item längerfristige Schadensereignisse\\
        \ra Verpflegung Einsatzkräfte \& Treibstoffversorgung\\
        \ra Löschmittelzuführung, Zuführung von Hilfsmitteln\\
        \ra Ablösung von Einsatzkräften \& Verständigung AG
    \end{itemize}
\end{sectionbox}