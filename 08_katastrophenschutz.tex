\section{Katastrophenschutz}
\begin{sectionbox}{Bay. Katastrophenschutzgesetz (BayKSG)}
    \begin{itemize}
        \item 24.07.96' mit gültiger Fassung vom 06.05.08'
        \item Zuweisung der KatS-Behörde\\
        \ra Aufgaben, Zuständigkeiten, Maßnahmen \& Befugnisse
        \item Ziele: vorbeugende \& abwehrende Maßnahmen gegen besondere Lagen, wie z.B. Natur, Gesundheit, Technik, Verkehr \& sonstige Großschadensereignisse
        \item KatS-Behörden: BSMI, Bezirksregierungen \& Kreisverwaltungsbehörden\\
        \ra incl. betroffene Ämter mit Spezialisierungen
        \item KatS-Behörde leitet Einsatz mit abgestimmten Maßnahmen\\
        \ra Weisungserteilung gegenüber jeder Behörde gleicher/niedriger Ebene + Einsatzkräften
    \end{itemize}
\end{sectionbox}
\begin{warningbox}{Definition: Katastrophe}
    \emph{klassische Schutzziele} := natürliche Lebensgrundlage oder bedeutende Sachwerte
    \begin{itemize} 
        \item Gefährdung/Bedrohung\\
        \ra von Leben/Gesundheit von vielen Menschen in ungewöhnlichen Ausmaß\\
        \ra oder klassischer Schutzziele
        \item Leitung der KatS-Behörde zwinged nötig\\
        \ra Abwehrung/Entstörung der Gefahr  
    \end{itemize}
    \begin{hintbox}{Vorbereitende Maßnahmen}
        \begin{itemize}
            \item Anlage KatS-Pläne, Einsatzpläne gegen besonderes Gefahrenpotenzial
            \item Regelung KatS-EL \& Führungsposition
            \item Organisatorische Vorkehrungen für Alarmierung
            \item Ausstattung der Einsatzleitung
            \item Durchführung von verpflichtenden KatS-Übungen
        \end{itemize}
    \end{hintbox}
    \begin{hintbox}{Feststellung Katastrophe durch KatS-Behörde}
        \begin{itemize}
            \item Festlegung Beginn \& Ende\\
            \ra unverzügliche Mitteilung der Öffentlichkeit
            \ra ebenso Aufsichtsbehörden\\
            \ra ebenso betroffene, benachbarten Behörden
        \end{itemize}
    \end{hintbox}
\end{warningbox}
\begin{sectionbox}{FüGK}
    \emph{Aufbau}: Leiter, personaler Dienst, Lage, Einsatz, Versorgung, Presse/Medien, Information/Kommunkation \& Fachberater/Verbindungspersonen
\end{sectionbox}
\subsection{Örtlicher Einsatzleiter (ÖEL)}
\begin{normbox}{\subsubsection{Allgemeines}}
    \begin{itemize}
        \item Benennung vorab der fachlich geeignete Person
        \item KatS-Behörde stellt am Schadensort ÖEL\\
        \ra Einsatzmaßnahmen/Weisungen vor Ort (6)
        \item Einsatzleitung vor Berufung\\
        \ra KatS-Behörde unverzüglich nachholen (6)
    \end{itemize}
\end{normbox}
\begin{normbox}{\subsubsection{Schadensereignisse unterhalb der KatS-Schwelle}}
    \begin{itemize}
        \item Kreisverwaltungsbehörde kann ÖEL bestellen\\
        \ra falls Erleichterung des geordneten Zusammenspiels
        \item Aufgaben der POL nach PAG bleiben unberührt (15)
    \end{itemize}
\end{normbox}
\begin{normbox}{\subsubsection{Mitwirkung im Katastrophenschutz}}
    \begin{itemize}
        \item Erfüllung nur dann, falls Gefährdung dringender eigener Aufgaben ausgeschlossen (7)
        \item Bestandteile
        \begin{itemize}
            \item Behörden \& Dienststellen
            \item Gemeinden, Landkreise  \& Bezirke
            \item Körperschaften, Anstalten \& Stiftungen des Freistaats
            \item Feuerwehren
            \item freiwillige Hilfsorganisationen
            \item Verbände der freien Wohlfahrtspflege 
        \end{itemize}
    \end{itemize}
\end{normbox}
\begin{warningbox}{Wie wird man zum ÖEL?}
    \begin{itemize}
        \item Bestellung zum ÖEL nach (6/15)\\
        \ra Einsatz als ÖEL: operativ-taktische Führung
        \item Alarmierung \& Einsatz FüGK\\
        \ra K-Fall: politisch-administrative Führung
    \end{itemize}
\end{warningbox}
\begin{normbox}{\subsubsection{Einsatzablauf}}
    \begin{itemize}
        \item Eintreffen an der Schadensstelle
        \item Lagefeststellung, Lagebeurteilung, Entschluss
        \item Rückmeldung an den Ansprechpartner FüGK
    \end{itemize}
\end{normbox}
\begin{normbox}{\subsubsection{Organisation}}
    Führungs-, Verbindungskräfte \& UG-ÖEL\\
    \ra POL, FF, RD, HTW, SanDienst, etc. 
\end{normbox}
\begin{normbox}{\subsubsection{Befugnisse}}
    \begin{itemize}
        \item operativ-taktische Führung \& Leitung Maßnahmen\\
        \ra Weisungsbefugnis gegenüber allen Einsatzkräften
        \item Inanspruchnahme Dritter (jegliche Art von Leistungen)
        \item Betretungsverbot, Platzverweis, Räumung Kat-Gebiet
    \end{itemize}
\end{normbox}
\begin{normbox}{\subsubsection{Aufgaben}}
    \begin{itemize}
        \item Errichtung Befehlsstelle ÖEL
        \item Erkundung Lage \& Planung Einsatz
        \item Warnung Bevölkerung
        \item Festlegung Einsatzabschnitte \& Bereitstellungsräume
        \item Führ-, Koordinations- \& Übewachungsfunktion
        \item Anforderung zusätzlicher Einsatzkräfte mit FüGK
        \item Herstellen \& Betreiben Fernmeldeverbindungen
        \item Rückmeldung KatS-Behörde über Lage \& Entwicklung
        \item Versorgung/Ablösung der Einsatzkräfte mit FüGK
    \end{itemize}
\end{normbox}
\begin{normbox}{\subsubsection{Aufgaben UG-ÖEL}}
    \begin{itemize}
        \item Errichtung ÖEL \& Kennzeichnung des Standortes
        \item Information FüGK
        \item Herstellung Kommunikationswege
        \item Unterstützung ÖEL: Lageerkundung, Einsatzplanung\\
        \ra Koordinierung eingesetzter Kräfte
        \item Führen Lagekarte \& Einsatztagebuch
    \end{itemize}
\end{normbox}
\begin{sectionbox}{Sanitätseinsatzleitung (SanEL)}
    \begin{itemize}
        \item besteht aus LNA und OrgL\\
        \ra LNA: medizinische Leitungsaufgaben\\
        \ra OrgL: taktische \& organisatorische Aufgaben
        \item allgemeine Aufgaben: geordnete Versorgung Verletzte\\
        \ra Verbindung ILS \& Zusammenarbeit Fachdienste  
        \item Aufgaben LNA: Leitung ärztlicher Einsatz\\
        \ra Feststellung \& Beurteilung aus medizinischer Sicht
        \item Aufgaben OrgL:  Leitung \& Koordination Einsatz\\
        \ra bestmögliche medizinische Versorgung Aller
        \item \emph{Aufgaben UG-SanEL}: Kommunikation, Registierung, Sichtung, Einrichtung von Einsatzabschnitten, Verletztentransport (Einweisung/Koordinierung)
    \end{itemize}
\end{sectionbox}
\begin{sectionbox}{Schnelleinsatzgruppen (SEG)}
    \begin{itemize}
        \item Behandlung: taktische Einheit für Verletzte\\
        \ra Führung dringendes benötigtes medizinisches Material
        \item Transport: Transport von Verletzten/Kranken\\
        \ra Übernahme von mind. 4 Verletzten/Kranken
        \item Betreuung: Versorgung \& Sicherung Betroffener\\
        \ra einfachste Verpflegung, Organisation Unterbringung sowie Basisnotfallnachsorge\\
        \ra max. Versorgung von 200 unverletzt Betroffene
        \item Verpflegung: Notküchen  für Betroffene/Einsatzkräfte
        \item Technik \& Sicherheit: Technik \& Logistik\\
        \ra enger Austausch Führungskräfte \& Belange UVV 
        \item Information \& Kommunikation: Unterstützung EL
        \item Rettungshunde: Personensuche (Amtshilfe)
        \item gefährliche Stoffe \& Güter: Zusammenarbeit FF\\
        \ra Dekontamination, Erstversorgung \& Rettung 
        \item Krad-Staffel: FR, Medikamententransport, Lotsendienst, Betreuung Reisende/Unglücksfälle/Stau 
    \end{itemize}
\end{sectionbox}
