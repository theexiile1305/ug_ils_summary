\section{Presse- und Medienarbeit}
\begin{sectionbox}{Allgemeines}
    \begin{itemize}
        \item Pressearbeit: Führungsaufgabe \& Investition
        \ra Förderung Glaubhaftigkeit, Transparenz \& Image\\
        \ra Erfolg erfordert positive öffentliche Meinung
        \item Medienarten: Print, Fundfunk \& neue Medien
    \end{itemize}
\end{sectionbox}
\begin{hintbox}{Pressemeldung}
    Zeitpunkt, Ort, Beteiligte, Sachverhalt, Zahlen, Rollen, mittelbare Folgen \& Impressum
\end{hintbox}
\begin{warningbox}{Beantwortung von Presseanfragen}
    Presseanfragen beantwortet von Leitstellenleitung bzw. zuständigen Schichtführern
\end{warningbox}
\begin{sectionbox}{Medien-Recht}
    \begin{itemize}
        \item Art. 1 GG: freie Meinungsäußerung \& ungerichtete Informationsfreiheit \ra keine Zensur\\
        \ra Gewährleistung via Rundfunk \& Film\\
        \item Art. 110 BV: freie Meinungsäußerung in Wort, Schrift, Druck, Bild \& sonstige Weise für Bewohner Bayerns
        \item Art. 111 BV: wahrheitsgemäße Berichterstattung der Presse über Vorgänge, Zustnände, Einrichtungen \& Personen im öffentlichen Leben
    \end{itemize}
\end{sectionbox}
\begin{sectionbox}{Auskunftsrecht - Art. 4 BayPresseG}
    \begin{itemize}
        \item Auskunftsrecht gegenüber Leitung d. Behörde
        \item Verweigerung, falls Verschwiegenheitspflicht
    \end{itemize}
\end{sectionbox}
\begin{sectionbox}{Presseausweis}
    \begin{itemize}
        \item nur nachweislich, hauptberuflich tätige Journalisten\\
        \ra öffentliches Interesse erforderlich
        \item Verwendung nur berufliche Zwecke
        \item Gültigkeit: 1 Jahr
    \end{itemize}
\end{sectionbox}