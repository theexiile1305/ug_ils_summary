\section{Aufgaben einer UG-ILS}
\begin{sectionbox}{Grundlagen / Aufgaben}
    \begin{itemize}
        \item Ausstattung: Regelbetrieb (Regelbesetzung), vergrößertes Aufkommen, Großschadenslagen
        \item mehrstufiges Verstärkungskonzept: 3-4 Disp.\\
        \ra I: + 1 Disp. mit Inhouse-Bereitschaft (5 min.)\\
        \ra II: + 1-2 Disp. mit Rufbereitschaft (30 min.)\\
        \ra III: UG-ILS \& dienstfreies Personal (30 min.)
    \end{itemize}
\end{sectionbox}
\begin{sectionbox}{Tätigkeit}
    \begin{itemize}
        \item Besetzung die AAPs \& Unterstützung bei der qualifizierten Notrufabfrage
        \item Unterstützung Disp. \& Weisung des Schichtführers
        \item Stellung UG: Über-/Unterordnungsverhältnis
    \end{itemize}
\end{sectionbox}
\subsection{Konsequenzen (Haftung/Strafrecht)}
\begin{normbox}{\subsubsection{Fahrlässigkeit}}
    \begin{itemize}
        \item Verwirklichung durch Tun oder Unterlassen: Schwerpunkt des Täterverhaltens/Vorwerfbarkeit\\
        \ra Abgrenzung irrelevant, wenn Garantenstellung 
        \item Garantenstellung bei Unterlassen aus Arbeitsvertrag und tatsächlicher Gewährsübernahme
        \item Verstoß gegen objekte/subjekte auferlegte Pflichten\\
        \ra Annahme von Hilfeersuchen\\
        \ra Erstellung und Bewertung von Meldebildern\\
        \ra Umsetzung des Meldebildes (Disposition)\\
        \ra Alarmierung der Einsatzkräfte\\
        \ra Feststellung und Beurteilung der Gesamtlage\\
        \ra Unterstützung der Einsatzleitung
        \item Tatbestandverwirklichung muss für Täter vorhersehbar und vermeidbar sein\\
        \ra generelle Voraussehbarkeit theoretischer Kausalverläufe nicht ausreichend\\
        \ra persönliche Fähigkeiten des Täters maßgebend
    \end{itemize}
\end{normbox}
\begin{normbox}{\subsubsection{Vorsatz}}
    \begin{itemize}
        \item Absicht: Täter strebt Erfolg an
        \item Wissen: Verwirklichung Tatbestand\\
        \ra bedingter Vorsatz: Täter nimmt Erfolg in Kauf
    \end{itemize}
\end{normbox}
\begin{hintbox}{Unterlassene Hilfeleistung}
    \begin{itemize}
        \item bedingter Vorsatz genügt
        \item kein Leisten von Hilfe bei Unglücksfällen/gemeiner Gefahr obwohl erforderlich und zuzumuten\\
        \ra ohne erhebliche eigene Gefahr\\
        \ra ohne Verletzung anderer wichtiger Pflichten
    \end{itemize}
\end{hintbox}
\begin{hintbox}{Fahrlässige Körperverletzung}
    Körperverletzung eines Anderen durch Fahrlässigkeit
\end{hintbox}
\begin{hintbox}{Fahrlässige Tötung}
    Verursachen des Todes eines Anderen durch Fahrlässigkeit
\end{hintbox}
\begin{normbox}{\subsubsection{Innerbetrieblicher Schadensausgleich}}
    \begin{itemize}
        \item Haftungsbeschränkung zum Schutze Mitarbeitenden\\
        \ra Vorsatz: volle Haftung\\
        \ra grobe Fahrlässigkeit:  volle Haftung, !Privatinsolvenz\\
        \ra mittlere Fahrlässigkeit: Aufteilung Schaden/Kosten\\
        \ra leichte Fahrlässigkeit: keine Haftung
    \end{itemize}
\end{normbox}
\begin{normbox}{\subsubsection{Haftpflichtversicherung}}
    \begin{itemize}
        \item Versicherung der Mitarbeitenden, falls kein Vorsatzoder grobe Fahrlässigkeit besteht
        \ra Übernahme des Anteils von Mitarbeitenden
    \end{itemize}
\end{normbox}
\begin{hintbox}{Vorsatzdelikt: Unterlassene Hilfeleistung}
    \begin{itemize}
        \item Vorraussetzungen\\
        \ra Unglücksfall: plöztliches Ereignis, erhebliche Gefahr für Individualsrechtsgut\\
        \ra Erforderlichkeit nach objekt naträglicher Prognose\\
        \ra Art/Maß der Hilfe richtet sich nach Fähigkeit/Möglichkeiten des Hilfspflichtigen
        \item Haftung: zivilrechtliche Haftung bei Beteiligten
        \item Vorraussetzungen: rechtswidrige Verletzung vermutet\\
        \ra Verschulden als Vorsatz oder Fahrlässigkeit
        \item Umfang: Dem Geschädigte so zu stellen, wie dieser im haftungsbegründeten Ereignis stünde
        \item Schuldner: in voller Höhe \& Gesamthaftung  
    \end{itemize}
\end{hintbox}